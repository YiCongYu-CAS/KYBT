\documentclass{standalone}
\usepackage{tikz}
\usepackage{graphicx} % 必须加载

\usepackage{newtxtext,newtxmath} % 推荐LaTeX数学字体方案
\tikzset{every node/.style={font=\rmfamily}} % 所有节点都用罗马体

\usetikzlibrary{positioning}
\usetikzlibrary{decorations.pathreplacing,decorations.markings}
\usetikzlibrary{patterns} % 加载 patterns 库
\usepackage{comment}

\usetikzlibrary{calc}


\colorlet{myred}{red!80!black}
\colorlet{myblue}{blue!80!black}
\colorlet{mygreen}{green!60!black}
\colorlet{myorange}{orange!70!red!60!black}
\colorlet{mydarkred}{red!30!black}
\colorlet{mydarkblue}{blue!40!black}
\colorlet{mydarkgreen}{green!30!black}
\colorlet{myred2}{red!50!white}

\colorlet{mylightblue}{blue!60!cyan!80!black!15}
\colorlet{mypurple}{blue!50!red!70}
\colorlet{gaugecol}{red!90!black!70} % Wiki red
\colorlet{leptoncol}{green!80!black!70} % Wiki green
\colorlet{quarkcol}{blue!85!cyan!95!black!55} % Wiki purple
\colorlet{quarkred}{red!98!black!55} % quark red
\colorlet{quarkblue}{blue!85!cyan!98!black!55} % quark blue
\colorlet{quarkgreen}{green!95!black!55} % quark green
\colorlet{gluoncyan}{cyan!100!black!55} % gluon cyan
\colorlet{gluongreen}{green!75!blue!95!black!70} % gluon green
\colorlet{gluonyellow}{yellow!98!black!55} % gluon yellow
\colorlet{gluonorange}{orange!100!black!65} % gluon orange
\colorlet{gluonmagenta}{magenta!100!black!70} % gluon magenta
\colorlet{scalarcol}{yellow!70!orange!98!black}
\colorlet{tensorcol}{blue!50!red!70} % Wiki light blue
\colorlet{groupcol}{orange!15}





\def\xs{2.5}
\def\ys{0.6}
\def\p{0.4}
\def\q{0.7}
\def\Height{2.8}
\def\Width{2.5}
\def\BW{3.3}
\def\De{-0.2}
\def\x{-0.74}
\def\y{1.83}
\def\xz{-2.86}
\def\yz{-1.83}
\def\De{0.1}

\begin{document}
\begin{tikzpicture}
  % 在 (0,0) 处插入图片,宽度设为3cm
  \node at (-1.8,0) {\includegraphics[width=5cm]{FigGrid.png}};
  %\node at (4,0) {\includegraphics[width=4cm]{FigSchematicFEM02.png}};


   \fill[pattern=north east lines, even odd rule]
        (\xz-\De,\yz-\De) rectangle (\x+\De,\y+\De)   % 外矩形
        (\xz,\yz) rectangle (\x,\y);  % 内矩形(挖空部分)
        
    % 可选:绘制辅助边框
    %\draw[thick] (0,0) rectangle (6,6);   % 外矩形边框
    \draw[thick] (\xz,\yz) rectangle (\x,\y);   % 内矩形边框

\coordinate (P1) at (\xz, \yz);
\coordinate (P3) at (\x, \y);

\begin{scope}
\clip (\xz,\yz) rectangle (\x, \y);
\draw[gray!60, line width=4.0, opacity=0.4] (P3) -- (P1);
\end{scope}

\coordinate (PS) at (1, -0.2);

\coordinate (P1) at ($(1.4, 1.7) + (PS)$);
\coordinate (P2) at ($(0.5,0) + (PS)$);
\coordinate (P3) at ($(2,0) + (PS)$);
\coordinate (P4) at ($(1.2,-1.4) + (PS)$);
\coordinate (P5) at ($(2.8,-1.6) + (PS)$);
\coordinate (P6) at ($(3.4,0) + (PS)$);
\coordinate (P7) at ($(2.7,1.9) + (PS)$);
\coordinate (PL) at ($(4.4,1.9) + (PS)$);

\coordinate (Q0) at ($(0,0) + (PS)$);
\coordinate (Q1) at ($(4,0) + (PS)$);


\draw[gray!60, line width=3.8, opacity=0.8] (Q0) -- (Q1);

\draw (P1) -- (P2) -- (P4) -- (P5) -- (P6) -- (P7) -- cycle;

\draw (P3) -- (P1);
%\draw (P3) -- (P2);
\draw (P3) -- (P4);
\draw (P3) -- (P5);
%\draw (P3) -- (P6);
\draw (P3) -- (P7);

\fill (P1) circle (1pt);
\fill (P2) circle (1pt);
\fill (P3) circle (1pt);
\fill (P4) circle (1pt);
\fill (P5) circle (1pt);
\fill (P6) circle (1pt);
\fill (P7) circle (1pt);

\node[above] at (P1) {$P_1$};
\node[above, xshift = -4] at (P2) {$P_2$};
\node[xshift = -8, yshift = 6] at (P3) {$P_3$};


%%%%%%%%%%%%%%%%%%%%%%%%%%%%%%%%%%%%%%%%%%%%%%%%%%%%%%%%%%%%%%%%%%%%%%%%%%%%%%%%%%%%%%%%%%%%%%%%%%%%%%%

%\fill (PL) circle (1pt);
\node at (PL) {};
 
\node at (-3.2, 2.2) {$(a)$};
\node at (0.7, 2.2) {$(b)$};
  
\end{tikzpicture}
\end{document}